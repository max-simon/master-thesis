\chapter{Introduction}\label{sec:introduction}

The \ac{ccs} is a highly productive coastal region \autocite{carr-2001-mostprod} that also drives productivity in the northeast Pacific Ocean \autocite{gruber-2011-eddy-red, nagai-2015-dom-role-meso, renault-2016-decoupling-prod, frischknecht-2018-3dpump}. Over the last years, various studies suggested that processes with scales of \SIrange{0.1}{10}{\kilo\metre} - the so called \textit{submesoscale} - could have significant impacts on upper ocean dynamics \autocite{thomas-2005-downfront-wind, boccaletti-2007, capet-2008-fronts1, sasaki-2014-mli-for-eddies, mcwilliams-2016-subm-currents, mahadevan-2016-subm-review, mcwilliams-2019-survey-subm, schubert-2020-subm-energy-cascade} and biological productivity \autocite{levy-2001-submesoscale-oligotrophic, lapeyre-2006-subm-vertical-pump, levy-2012-physics-to-life, omand-2015-poc-export, mahadevan-2016-subm-review, stukel-2017-carbon-export, levy-2018-role-structuring, kessouri-2020-seasonal-prod}. However, the question about consequences for a consideration in numerical models remains open \autocite{fox-kemper-2019-challenges}. Therefore, this study examines how submesoscale processes impact larger scales and how this alter biological productivity.\\
\\
The \ac{ccs} is located at the west coast of North America. In this study, the focus is on the central \ac{ccs} which extends from Cape Mendocino at $\SI{40}{\degree}$N to Point Conception at $\SI{35}{\degree}$N \autocite{checkley-2009-currentstructure-calcs}. During spring and summer, equatorward winds lead to offshore Ekman transport and upwelling of cold, nutrient-rich water at the coast which fuels biological productivity \autocite{huyer-1983-upwelling}. In addition, two large scale flows occur in the region. The \ac{cc} is part of the North Pacific Gyre and appears as a broad, equatorward flow with velocities of $\around \SI{0.1}{\metre\per\second}$. It is accompanied by intense jets of $\around \SI{0.5}{\metre\per\second}$ which are $\around \SI{75}{\kilo\metre}$ wide \autocite{huyer-1991-jet}. The poleward \ac{cuc} develops in response to equatorward winds. It is located close to the coast at \SIrange{100}{300}{\metre} depth \autocite{mccreary-1987-dynamics-calcs}. Biological and physical processes are tightly coupled in the \ac{ccs} \autocite{bograd-2001-physbiocoupling} and are subject to strong interannual variability driven but not limited by basin-scale effects \autocite{frischknecht-2015-remote}. As a representative of \acp{ebus}, the \ac{ccs} shares similarities in ocean dynamics and biological productivity with other \ac{ebus} \autocite{chavez-2009-ebus}.\\
\\
The highly dynamic regime in the \ac{ccs} gives rise to mesoscale processes, i.e. mesoscale eddies \autocite{kelly-1998-eke-obs}. Emerging from instabilities in large scale currents, these coherent vortices have radii of \SIrange{35}{100}{\kilo\metre} (first Rossby radius of deformation) and lifetimes of up to several months \autocite{kurian-2011-eddy-props}. Mesoscale eddies are vertically hydrostatic and have a geostrophic horizontal flow, that is a counter-clockwise rotation of cyclones around a positive density anomaly in the Northern Hemisphere (anticyclones clockwise around a negative density anomaly) \autocite{mcwilliams-2008-nature-eddies}. Due to the vertical displacement of isopycnals, mesoscale eddies are also associated with an anomaly in \ac{ssh} that is negative for cyclones and positive for anticyclones. This allows for tracking of mesoscale eddies with satellites \autocite{ducet-2000-first-ssh, chelton-2011}.\\
\\
Mesoscale eddies impact biological productivity in several ways. They stir surrounding waters \autocite{rossi-2008-stirring, chelton-2011-surfacechl}, trap fluids present during formation and thereby transport biogeochemical properties to distinct regimes \autocite{flierl-1981-trapping-theo, chelton-2011-surfacechl, nagai-2015-dom-role-meso}. In addition to lateral transport mechanisms, the vertical displacement of isopycnals impacts biological productivity. The displacement allows for adiabatic fluxes (along isopycnals) of biogeochemical tracers in to or out of the euphotic zone where they can fuel productivity \autocite{falkowski-1991-eddy-pumping, freilich-2019-w-decomp}. Further, the displacement also changes the \ac{mld} which is decreased for cyclones and increased for anticyclones. Thereby, the light exposure time of phytoplankton in light limited regimes is affected \autocite{mcgilli-2016-meso-review}. Moreover, vertical transport of nutrients or biomass can be induced by surface stress (interaction with wind) \autocite{stern-1965-eddy-wind, mcgilli-2008-response, gaube-2015-eddy-wind} or vortex deformations (interaction with other eddies or currents) \autocite{martin-2001-vert-vel-mesoeddies}. All these effects have been extensively investigated in observations \autocite{friedrichs-2009-calcs-npp-obs-sat, kahru-2009-calcs-npp-obs} and numerical models \autocite{nagai-2015-dom-role-meso, mcgilli-2016-meso-review, frischknecht-2018-3dpump}.\\
\\
In contrast, submesoscale motions are often not resolved in numerical models and less is known about their interaction with other scales and processes. The submesoscale lies between the mesoscale (dominated by rotation of the earth) and vertical turbulent mixing. It comprises spatial scales of $\SI{0.1}{\kilo\metre}$ to $\SI{10}{\kilo\metre}$ and temporal scales of days to weeks, which makes observations and numerical modelling challenging \autocite{thomas-2008-subm}. A lot of processes fall into the submesoscale (e.g. subsurface coherent vortices \autocite{mcwilliams-1985-scv} or currents in the deep sea \autocite{vic-2018-dispersion-deep-sea}), but the focus of this study is on submesoscale fronts in the upper ocean and associated processes.\\
In the upper ocean, submesoscale fronts emerge at horizontal density gradients and are powered by atmospheric forcing and mesoscale strain \autocite{thomas-2008-subm}. Further, they can be intensified by Ekman transport induced by down-front winds \autocite{thomas-2005-downfront-wind}. They appear as transitory fronts of strong vertical velocity ($\around \SI{e-4}{\metre\per\second}$, always up- and downwelling paired), are $\around \SI{10}{\kilo\metre}$ wide and cover large parts of the upper ocean \autocite{capet-2008-fronts2}. Their penetration depth is modulated by the \ac{mld}, thus they are strongest in winter \autocite{mensa-2013-seasonality-mli, callies-2015-seasonality-subm}. An illustrative description of the physical processes leading to the formation of fronts - also called \textit{frontogenesis} - can be found in \textcite{mahadevan-2016-subm-review}. Because the submesoscale is close to the resolution limit of current numerical models,

\begin{hypothesis}
we hypothesize that submesoscale frontogenesis is better resolved with higher horizontal resolution.\label{hypo:h1}
\end{hypothesis}

Submesoscale fronts are believed to impact the mesoscale. On the one hand, instabilities, also called \ac{mli}, can detach as \ac{mle} from frontal regions. The horizontal scale of \ac{mle} can be as small as $\around \SI{5}{\kilo\metre}$ but they grow with time \autocite{boccaletti-2007}. \ac{mle} restratify the mixed layer by overturning isopycnals from vertical to horizontal \autocite{foxkemper-2008-mle-param, levy-2012-physics-to-life, whitt-2017-ml-stratification}. Further, small eddies like \ac{mle} can be absorbed by mesoscale eddies and thus increases the kinetic energy of the merged eddy. Thereby, kinetic energy is transported from smaller to larger scales which is called \textit{inverse energy cascade} \autocite{sasaki-2014-mli-for-eddies, schubert-2020-subm-energy-cascade}. On the other hand, symmetric instabilities in frontal regions dissipate kinetic energy and thereby drive a forward energy cascade from mesoscale to turbulent mixing \autocite{dasaro-subm-dissipation-obs, schubert-2020-subm-energy-cascade}.\\
\\
The question arises if submesoscale fronts have to be considered in numerical models, despite the increased computational resources this requires. \textcite{capet-2008-fronts1} were one of the first to address this question in the context of the \ac{ccs}. They compared numerical models of an idealized \ac{ccs} with different horizontal resolutions (down to $\SI{750}{\metre}$) and investigated the dynamics at submesoscale fronts \autocite{capet-2008-fronts2} as well as the corresponding energy cascade \autocite{capet-2008-fronts3}. In particular, they determined the length scale at which the energy flux changes sign (from inverse cascade to forward cascade) to be $\around \SI{35}{\kilo\metre}$ \autocite{capet-2008-fronts3} which is much smaller than estimates from altimetry data of $\around \SI{100}{\kilo\metre}$ \autocite{tulloch-2011-inverse-altimeter}. This scale is impacted by eddies with a diameter as small as $\around \SI{17}{\kilo\metre}$ \autocite[Appendix B]{schubert-2020-subm-energy-cascade} which is close to the length scale of \ac{mle}. Recently, \textcite{schubert-2020-subm-energy-cascade} confirmed that \ac{mle} strongly support the inverse energy cascade and that kinetic energy at mesoscale is reduced by up to $\SI{20}{\percent}$ when submesoscale motions are not resolved. However, the impact on single eddies remains unclear.\\
\\
There are only few studies on how mesoscale eddy properties change when submesoscale fronts are resolved. In general, anticyclonic vorticity facilitate submesoscale instabilities \autocite{haine-1998-mli, mcwilliams-2004-aai, thomas-2013-classification-sym-inst}. This was observed for an idealized setup by \textcite{brannigan-2017} who found instabilities to grow faster and at coarser resolutions in anticyclones than in cyclones. Furthermore, \textcite{schubert-2019-agulhas} compared numerical models of the Agulhas system with different horizontal resolutions and found that especially cyclones with a large amplitude in their \ac{ssh} anomaly are better resolved in the high-resolution model. In this study, the focus is on the density anomaly of mesoscale eddies which is linked to the \ac{ssh} anomaly. It describes the vertical displacement of isopycnals in the eddy core and thereby impacts biological properties. Following the findings of \textcite{schubert-2019-agulhas},

\begin{hypothesis}
we hypothesize that the resolution of submesoscale fronts leads to a strengthening of the density anomaly in cyclones whereas it remains unchanged for anticyclones.\label{hypo:h2}
\end{hypothesis}

The question how submesoscale fronts impact biological productivity is a much debated issue. On the one hand, the strong vertical velocities associated with submesoscale fronts can enhance the vertical transport of nutrients and organic matter with consequences for productivity and diversity \autocite{levy-2018-role-structuring, whitt-2019-subm-flux-storm}. On the other hand, \textcite{levy-2018-role-structuring} concluded that their impact on nutrient supply to the euphotic zone is limited. The reason is that the fronts are bound to the \ac{mld} and that their seasonality is out of phase with the biological productive season in summer. However, they point out that restratification induced by \ac{mle} can impact productivity in light limited regimes (i.e. spring blooms) and that submesoscale dynamics might affect biodiversity, community structure and patchiness \autocite{levy-2018-role-structuring}. Nevertheless, resolving submesoscale processes can alter biological productivity indirectly, e.g. due to changes in the mean circulation \autocite{levy-2012-large-scale-impacts}. Further, mesoscale processes which transport nutrients and organic matter, can be impacted as well. This was observed recently by \textcite{kessouri-2020-seasonal-prod}. They compared nutrient fluxes and biological productivity in the \ac{ccs} between models with $\SI{4}{\kilo\metre}$ and $\SI{1}{\kilo\metre}$ horizontal resolution and found that eddy quenching (an eddy-induced reduction of primary productivity at the coast, see \autoref{sec:npp-mechanisms}) is intensified in the high-resolution model. These findings and the changes induced by an increase of horizontal resolution are revisited in this study.

\begin{hypothesis}
We hypothesize that the biological productive band at the coast is broadened for an increased horizontal resolution.\label{hypo:h3}
\end{hypothesis}

The study is based on the comparison of two numerical models of the \ac{ccs}: a mid-resolution (conventional) model with $\SI{7.0}{\kilo\metre}$ horizontal resolution and a high-resolution (front-permitting) model with $\SI{2.8}{\kilo\metre}$ horizontal resolution (see \autoref{sec:data-methods-model}). First, mesoscale eddies are identified using a \ac{ssh}-based detection algorithm (see \autoref{sec:data-methods-eddydetection}) and compared between the two models in \autoref{sec:mesoscale}. The presence and characteristics of submesoscale fronts are examined in \autoref{sec:subm}. To this end, a novel detection algorithm was developed which is described in detail in \autoref{sec:data-methods-submdet}. Thereafter, the differences between the two models regarding biological productivity, i.e. \ac{npp}, are explored in \autoref{sec:npp}. In \autoref{sec:discussion} the results are gathered and implications are discussed.