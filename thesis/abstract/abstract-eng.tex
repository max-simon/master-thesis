Submesoscale motions are often not resolved in numerical models, although recent studies suggest that they interact with mesoscale processes. This might be particularly relevant for regions like the California Current System (CCS) where mesoscale processes redistribute nutrients and organic matter to offshore regions. In this study, the impact of submesoscale fronts on mesoscale eddies and biological productivity is examined by comparing two models of the CCS with different horizontal resolutions: a conventional ($\SI{7.0}{\kilo\metre}$) and a front-permitting resolution ($\SI{2.8}{\kilo\metre}$). A novel detection algorithm was developed which allows quantifying the area covered by submesoscale fronts. The algorithm reveals that fronts occur more often in anticyclones than in cyclones. This results in a weakening of the density anomaly associated with anticyclones by $\SI{40}{\percent}$ during winter for the increased resolution. Further, the energy cascade of mesoscale eddies is better resolved contributing to the seasonal evolution of eddy kinetic energy. Finally, the biological productive band at the coast broadens, presumably driven by enhanced lateral transport of nutrients. The results demonstrate that submesoscale and mesoscale motions are inextricably linked and that regional numerical models should aim to resolve submesoscale fronts for future studies.