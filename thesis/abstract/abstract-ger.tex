Numerische Modelle lösen submesoskalige Prozesse häufig nicht auf, obwohl gezeigt wurde, dass sie auch mesoskalige Prozesse beeinflussen. Dies ist besonders in Regionen wie dem \textit{California Current System} (CCS) relevant, da dort mesoskalige Prozesse am Transport von Nährstoffen und organischem Material in küstenferne Regionen beteiligt sind. In dieser Arbeit wird der Einfluss von submesoskaligen Fronten auf mesoskalige Eddies und biologische Produktivität untersucht, indem zwei Modelle des CCS mit unterschiedlicher horizontaler Auflösung verglichen werden: eine konventionelle ($\SI{7.0}{\kilo\metre}$) und eine feine Auflösung ($\SI{2.8}{\kilo\metre}$). Zudem wurde ein Erkennungsalgorithmus für submesoskalige Fronten entwickelt, der deren eingenommene Fläche ermittelt. Der Algorithmus zeigt auf, dass Fronten häufiger in Anticyclonen als in Cyclonen auftreten. Dies führt in der feineren Auflösung zu einer Abschwächung der Dichteanomalie in Anticyclonen von $\SI{40}{\percent}$ im Winter. Zudem wird die Energiekaskade von mesoskaligen Eddies und damit auch die Saisonalität der \textit{eddy kinetic energy} (EKE) besser aufgelöst. Außerdem verbreitert sich die biologisch produktive Zone an der Küste, vermutlich durch verstärkten lateralen Transport von Nährstoffen. Die Ergebnisse zeigen, wie eng submesoskalige und mesoskalige Prozesse miteinander verbunden sind und dass regionale numerische Modelle submesoskalige Fronten für kommende Studien auflösen sollten. 