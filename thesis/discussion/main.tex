\chapter{Discussion \& Outlook}\label{sec:discussion}

The study addressed the impact of submesoscale fronts on mesoscale eddies and biological productivity. To this end, two models of the \ac{ccs} with different horizontal resolutions were compared, a conventional and a front-permitting resolution. First, effects on mesoscale eddies were examined by analysing the properties and strength of eddies (\autoref{sec:mesoscale}). Afterwards, submesoscale fronts were characterized and analyzed for their interaction with mesoscale eddies. This was done using a novel detection algorithm for submesoscale fronts in the vertical velocity field (\autoref{sec:subm}). Finally, the lateral distribution of biological productivity was compared between the two models (\autoref{sec:npp}). In the following, the results are discussed with respect to the initial hypotheses.\\
\\
With respect to \hypothesisref{hypo:h1}, it was found that submesoscale frontogenesis is much better resolved in \ac{hr} than in \ac{mr}. In \ac{hr}, submesoscale fronts are ubiquitous during winter and spring, cover large parts of the upper ocean and shape the vertical velocity structure in the mixed layer. Important characteristics of frontogenesis, e.g. the seasonality, penetration depth or asymmetry of up- and downwelling, are confirmed by the detection algorithm for submesoscale fronts. This can also be understood as a validation for the algorithm.\\
\\
The increase in resolution leads to a reduction of \ac{npp} in the nearshore region by $\around \SI{4}{\percent}$. Because mesoscale eddies are more energetic in \ac{hr}, they are more effective in trapping water masses and hence exporting nutrients and organic matter to offshore regions. Therefore, the reduction of \ac{npp} at the coast in \ac{hr} can be attributed to enhanced eddy quenching. Also \textcite{kessouri-2020-seasonal-prod} found a reduction of nearshore \ac{npp} and attributed it to enhanced eddy quenching. However, the reduction in their high-resolution model appears later than in \ac{hr}. A reason for this delay could be different strengths of \ac{cc} and \ac{cuc} which drive eddy formation at the coast. Moreover, \textcite{kessouri-2020-seasonal-prod} used a different atmospheric forcing which also impacts the timing of biological productivity. Nevertheless, the results suggest that an increase in horizontal resolution enhances eddy quenching and decreases biological productivity at the coast. Albeit, Lagrangian experiments are required to confirm the exact mechanism involved in this reduction.\\
\\
In the offshore region, \ac{npp} is increased in \ac{hr} by $\around \SI{6}{\percent}$, especially during spring. During this time, biological productivity is highest in the offshore region for both models. The cause of this seasonality can be summarized by two effects \autocite{mahadevan-2012-na-spring-bloom}. On the one hand, productivity is low during winter and nutrients can accumulate in the deep mixed layer. On the other hand, the increased solar radiation and decreased buoyancy forcing in spring cause restratification of the surface layer and thereby enhances light exposure time. This leads to an increase in productivity (spring bloom) and consumption of the nutrients \autocite{mahadevan-2012-na-spring-bloom}. Submesoscale fronts and associated \ac{mle} support restratification and can thereby enhance productivity \autocite{mahadevan-2012-na-spring-bloom} and vertical transport of nutrients \autocite{mahadevan-2016-subm-review}. Because submesoscale frontogenesis is much better resolved in \ac{hr}, the increase in \ac{npp} in \ac{hr} during spring is therefore reasonable.\\
However, there is a discrepancy to results of \textcite{kessouri-2020-seasonal-prod} who found a year-round increase of \ac{npp} in the offshore region for their high-resolution model. As mentioned above, offshore productivity is light limited during spring and not driven by coastal processes \autocite{longhurst-2007-empfehlung-urs}. In this regime, submesoscale motions can enhance productivity as soon as they are resolved in the model. In contrast, an enhancement of offshore productivity in a nutrient limited regime (i.e. during summer and autumn) does only work in conjunction with lateral transport of nutrients from the coast \autocite{lathuliere-2010, gruber-2011-eddy-red}. Also \textcite{kessouri-2020-seasonal-prod} ascribe the increase in offshore \ac{npp} to a combination of both, an increased nutrient subduction at the coast and a resupply by submesoscale fronts in the offshore region. As discussed, it is very likely that the impact of submesoscale motions on these transport processes is not fully captured in the present study as the nearshore reduction of \ac{npp} has not converged at the end of integration in \ac{hr}. Therefore, the observed discrepancy to results of \textcite{kessouri-2020-seasonal-prod} is presumably be a result of the short integration time in \ac{hr}.\\
\\
Overall, the increase in resolution leads to a reduction of \ac{npp} at the coast and to an increase in the offshore region. Therefore, the productive band broadens in \ac{hr} which was stated in \hypothesisref{hypo:h3}. However, the results should be validated with a longer integration time.\\
\\
Finally, consequences for mesoscale eddies are discussed. Compared to \ac{mr}, the \ac{eke} of mesoscale eddies in the offshore region is increased by up to $\SI{50}{\percent}$ in \ac{hr}. As described in \autoref{sec:introduction}, the inverse energy cascade of mesoscale eddies is driven by the absorption of eddies as small as $\SI{17}{\kilo\metre}$ and the cascaded energy reaches the mesoscale in late spring or early summer \autocite{capet-2008-fronts3, schubert-2020-subm-energy-cascade}. This matches the results for \ac{hr}: small eddies and submesoscale fronts are much better resolved in \ac{hr} and the increase in \ac{eke} is especially strong during early summer. Therefore, the higher \ac{eke} in \ac{hr} can be attributed to a more complete representation of the inverse energy cascade of mesoscale eddies.\\
\\
By contrast, the density anomaly of mesoscale eddies is impacted in a different way. On the one hand, offshore cyclones reveal a seasonality of the density anomaly in \ac{hr} which is similar to the trend in \ac{eke}. Though, the overall magnitude changes only little and is not intensified as it was hypothesized. On the other hand, offshore anticyclones also obtain a seasonality of the density anomaly in \ac{hr} which is similar to that of cyclones. However, the anomaly is severely weaker in \ac{hr} than in \ac{mr}. During winter, when the weakening is strongest, the density anomaly is reduced by $\around \SI{40}{\percent}$. Therefore, \hypothesisref{hypo:h2} has to be modified such that submesoscale fronts mainly weaken the density anomaly in mesoscale anticyclones, whereas cyclones are less affected. This does not contradict the results of \textcite{schubert-2019-agulhas} which motivated this hypothesis, because they only examined the \ac{ssh} anomaly and not the density anomaly.\\
\\
The surprisingly strong weakening of offshore anticyclones in \ac{hr} does not only appear in the reduced density anomaly. Also the increase in \ac{eke} associated with anticyclones during winter is, compared to the increase observed for cyclones, very weak. In addition, the enhanced asymmetry between cyclonic and anticyclonic eddies in \ac{hr} indicates a weakening of anticyclones.\\
There are several mechanisms which can cause this weakening. First, symmetric instabilities (which drive a forward energy cascade) are pronounced in anticyclones because of the negative vorticity \autocite{thomas-2013-classification-sym-inst}. \Textcite{brannigan-2017} found that instabilities grow in an anticyclone already at $\SI{2}{\kilo\metre}$ horizontal resolution (similar to \ac{hr}), whereas a resolution of $\SI{0.25}{\kilo\metre}$ is required to observe such instabilities in cyclones. Secondly, the detection algorithm for submesoscale fronts revealed that fronts occur more often in anticyclones than in cyclones. The fronts do not only drive a forward energy cascade \autocite{dasaro-subm-dissipation-obs}, but also cause a positive vertical heat flux from deeper and colder waters to warm waters near the surface \autocite{klein-2019-review-subm}. This can further undermine the positive temperature anomaly of anticyclones \autocite{frenger-2015-so-eddy-phen}. However, the reason for the enhanced presence of submesoscale fronts in anticyclones remains unclear. One explanation might be the negative vorticity which promotes instabilities \autocite{thomas-2013-classification-sym-inst}. Another reason could be that the \ac{mld} in anticyclones is deeper than in cyclones which favors frontogenesis there.\\
Considering that submesoscale fronts already appear at the lower limit of horizontal resolution in \ac{hr}, symmetric instabilities presumably contribute only little to the observed weakening of anticyclones. Instead, we speculate that the heat flux driven by submesoscale fronts is the main cause. Yet, this has to be investigated more systematically in future studies.

\subsubsection{Conclusion}

There are three important conclusions that can be drawn from the presented results. First, submesoscale fronts shape vertical velocities in the mixed layer. Their omnipresence as well as the strong vertical velocities turns them into important components for upper ocean dynamics. Secondly, submesoscale fronts strongly interact with larger scales, especially with mesoscale eddies. The fronts and \ac{mle} fuel an inverse energy cascade, contribute to the seasonality of eddies and also impact their density anomalies. The latter is especially strong in anticyclones and contributes to the dominance of cyclonic polarity. Finally, biological productivity is indeed impacted by the increase in horizontal resolution. \Textcite{levy-2018-role-structuring} claimed that the impact of submesoscale fronts on biological productivity is only little because their seasonality is out of phase with biological productivity. However, this should be mitigated for regions where biological productivity is also shaped by mesoscale transport processes. In such regions, to which \acp{ebus} and the \ac{ccs} belong to, productivity is also affected by changes in the mesoscale driven by interactions with submesoscale fronts.\\
The study demonstrated that submesoscale and mesoscale processes are inextricably linked and that they have to be treated in conjunction with each other. Therefore, numerical models for regional studies should parameterize the discussed effects or deploy a horizontal resolution of $\mathcal{O}(\SI{1}{\kilo\metre})$, even if the study is not focused on submesoscale processes.

\subsubsection{Outlook}

The study has several shortcomings which should be addressed in future studies. First, the model should be tuned with special attention to the strength of \ac{cc} and \ac{cuc} as well as to a more realistic productivity. Furthermore, the models should be run with a longer integration time to capture also long-term effects. Moreover, it would be beneficial to adjust the atmospheric forcing to a realistic forcing to address the interannual variability in the region. These changes lead to a more realistic representation of the \ac{ccs} and presumably to a better comparison to observational data.\\
\\
As mentioned before, the reason for the enhanced presence of submesoscale fronts in anticyclones as well as the exact mechanism for the weakening remain unclear. This should be addressed more systematically in a follow-up study. The developed detection algorithm for submesoscale fronts might be useful for this, but the thresholding step should be improved to reduce the sensitivity to the set of parameters.\\
\\
Future studies should also consider the interaction of submesoscale motions and coastal filaments. These cold water filaments have sharp density gradients to the surrounding water and play an important role in the lateral export of nutrients \autocite{nagai-2015-dom-role-meso}. Further, submesoscale coherent vortices which encapsulate subsurface water and transport it to offshore regions \autocite{frenger-2018-puddies}, should be included as well. Also inertial gravity waves which interact with mesoscale motions \autocite{klein-2019-review-subm} should be considered. Regarding biological productivity, the proposed impacts of submesoscale fronts should be validated in Lagrangian experiments with productive particles. Furthermore, it was shown that submesoscale motions also impact carbon export \autocite{omand-2015-poc-export, stukel-2017-carbon-export} and biodiversity \autocite{levy-2018-role-structuring}. These effects can be addressed with the used setup as well. Finally, the applicability of the presented results to other \ac{ebus} or other regions should be examined.